\chapter{Podsumowanie}
\label{cha:podsumowanie}

Celem pracy był przegląd metod lokalnego przechowywania danych z użyciem przeglądarki internetowej oraz projekt i implementacja aplikacji wspierającej tryby offline oraz online przy użyciu HTML5. Cel pracy został w pełni zrealizowany.

Wszystkie aktualnie dostępne metody lokalnego przechowywania danych zostały kompleksowo przetestowane oraz opisane. Szczególną uwagę zwrócono na wsparcie przeglądarek, wydajność oraz użyteczność danej metody. Metody lokalnego przechowywania danych cechują się dynamicznym rozwojem, dlatego też każda z metod została oceniona pod względem wsparcia oraz nowych funkcjonalności w nadchodzących wersjach przeglądarek internetowych.

Aplikacja OffCalendar została stworzona z naciskiem na pracę w trybie offline. Brak połączenia internetowego nie powoduje utraty możliwości zarządzania wydarzeniami, co jest szczególnie ważne przy coraz większym udziale urządzeń mobilnych. Responsywny interfejs użytkownika odpowiada za wysoką użyteczność niezależnie od rozmiaru urządzenia docelowego, natomiast komunikacja bazująca na wzorcu architektury REST oraz standardzie JSON pozwala na minimalizację przesyłanych danych.

Projekt aplikacji OffCalendar może być rozwijany w wielu kierunkach. Od strony funkcjonalności jest to implementacja wydarzeń grupowych oraz wydarzeń cyklicznych. Ze strony technologii jest to zapewnienie odpowiedniej skalowalności aplikacji, implementacja widoków dedykowanych dla druku oraz zapewnienie wsparcia geolokacji dla wydarzeń odbywających się w określonym miejscu.