\chapter{Podsumowanie}
\label{cha:podsumowanie}

Celem pracy był przegląd metod przechowywania danych przy użyciu przeglądarki internetowej oraz ich analiza pod kątem kompatybilności oraz wydajności w określonych klasach problemów. Kolejnym celem pracy był projekt oraz implementacja aplikacji wspierającej tryby offline oraz online w oparciu o technologię HTML5 . Cele pracy zostały w pełni zrealizowane.

Wszystkie aktualnie dostępne metody lokalnego przechowywania danych zostały kompleksowo przetestowane oraz opisane. Szczególną uwagę zwrócono na wsparcie przeglądarek, wydajność oraz użyteczność danej metody. Metody lokalnego przechowywania danych cechują się dynamicznym rozwojem, dlatego też każda z metod została oceniona pod względem wsparcia oraz nowych funkcjonalności w nadchodzących wersjach przeglądarek internetowych.

Na podstawie wykonanej analizy można stwierdzić, że metody WebSQL oraz Filesystem API nie oferują wystarczającego wsparcia wśród popularnych przeglądarek internetowych aby mogły zostać z powodzeniem wykorzystane w komercyjnych projektach. Ze względu na zawieszenie prac nad ich rozwojem przez organizację W3C metody te nie będą również posiadać wsparcia w nadchodzących wersjach przeglądarek internetowych. Ze względu na unikalne względem innych metod własności, WebSQL oraz Filesystem API mogą stanowić dobry wybór dla projektów niekomercyjnych, w których możliwe jest narzucenie używania przeglądarki wspierającej daną metodę.

Zupełnie inaczej sytuacja wygląda dla metod IndexedDB oraz Web Storage. Specyfikacje obu rozwiązań są na bieżąco rozwijane przez organizację W3C. Obie metody oferują wysoki poziom wsparcia w obecnych wersjach popularnych przeglądarek internetowych, natomiast przeglądarki które nie oferują wsparcia w aktualnej wersji, będą posiadać je w wersjach nadchodzących. Cechy te sprawiają, że metody IndexedDB oraz Web Storage mogą być z powodzeniem stosowane w aplikacjach zarówno publicznych jak i prywatnych.

Metoda Web Storage nie posiada możliwości przechowywania zaawansowanych struktur danych. Z tego względu, jej wartość najlepiej widoczna jest w projektach, które wymagają przechowywania niewielkich ilości informacji mogących być w szybki i wydajny sposób modyfikowanych. Pliki cookie są przesyłane przy każdym zapytaniu HTTP kierowanym do serwera, co w wielu przypadkach nie jest wymagane do poprawnego działania danej aplikacji. Przesłanie danych przechowywanych za pomocą metody Web Storage jest opcjonalne, co czyni ją w tych przypadkach wydajniejszą alternatywą.

IndexedDB jest metodą doskonale sprawdzającą się w rozwiązaniach wymagających przechowywania oraz zarządzania znaczną ilością rozbudowanych struktur danych. Ze względu na możliwość tworzenia indeksów metoda ta oferuje wysoką wydajność w przypadku aplikacji wymagających szybkiego filtrowania rekordów. IndexedDB może być stosowana w aplikacjach wymagających przechowywania par klucz-wartość, jednak ze względu na skomplikowany interfejs dostępu do danych oraz asynchroniczność zapytań, dużo lepszym rozwiązaniem jest tutaj metoda Web Storage.

Aplikacja OffCalendar została stworzona z naciskiem na pracę w trybie offline. Brak połączenia internetowego nie powoduje utraty możliwości zarządzania wydarzeniami, co jest szczególnie ważne przy coraz większym udziale urządzeń mobilnych. Responsywny interfejs użytkownika odpowiada za wysoką użyteczność niezależnie od rozmiaru urządzenia docelowego, natomiast komunikacja bazująca na wzorcu architektury REST oraz standardzie JSON pozwala na minimalizację przesyłanych danych.

OffCalendar używa dwóch metod lokalnego przechowywania danych: IndexedDB oraz Web Storage. Metoda IndexedDB została użyta do lokalnego zarządzania wydarzeniami użytkowników. Zastosowanie indeksów na wybranych atrybutach wydarzeń pozwoliło na wydajne wyszukiwanie oraz synchronizację wydarzeń. Ze względu na szybkość i łatwość dostępu do danych, metoda Web Storage została wykorzystana do przechowywania danych użytkownika oraz stempli czasowych używanych w procesie synchronizacji.

Projekt aplikacji OffCalendar może być rozwijany w wielu kierunkach. Od strony funkcjonalności jest to implementacja wydarzeń grupowych oraz wydarzeń cyklicznych. Ze strony technologii jest to zapewnienie odpowiedniej skalowalności aplikacji, implementacja widoków dedykowanych dla druku oraz zapewnienie wsparcia geolokacji dla wydarzeń odbywających się w określonym miejscu.