\chapter{Implementacja}
\label{cha:impl}

Rozdział zawiera odwzorowanie konceptu przedstawionego w rozdziale trzecim na kod aplikacji \mbox{OffCalendar}, zarówno po stronie klienta, jak i serwera. 

Prześledzono technologie i biblioteki użyte podczas tworzenia interfejsu graficznego, mechanizmy odpowiedzialne za manipulację wydarzeniami, przedstawiono rozwiązanie kwestii automatycznej detekcji stanu połączenia internetowego oraz synchronizacji danych. Opisano również komunikację z bazą danych oraz implementację bezstanowych interfejsów niezbędnych do zachowania integralności części klienckiej oraz serwerowej.

Zastosowane rozwiązania zostaną poparte szczegółowymi objaśnieniami w postaci kodu oraz wzbogacone grafikami i schematami przedstawiającymi wygląd aplikacji docelowej.


\section{Aplikacja kliencka}
\label{sec:apKli}



\subsection{Interfejs graficzny}
\label{sec:intGraf}



\subsection{Zarządzanie wydarzeniami}
\label{sec:zarzWyd}



\section{Detekcja połączenia internetowego}
\label{sec:detPolInt}



\subsection{Automatyczna synchronizacja danych}
\label{autSynDanych}


\section{Aplikacja serwerowa}
\label{sec:apSerw}



\subsection{Komunikacja z bazą danych}
\label{komBazaDanych}



\subsection{Bezstanowe interfejsy}
\label{bezstInter}