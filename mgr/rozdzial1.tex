\chapter{Wstęp}
\label{cha:wstep}

Pozycja sieci Internet jako podstawowego i powszechnego medium komunikacyjnego jest niezachwiana od momentu jej powstania. Obecnie niemal wszystkie aplikacje dokonują synchronizacji za pomocą protokołów internetowych bez względu na platformę, na którą są przeznaczone. Dynamicznie rozwijające się standardy związane z protokołem HTTP (Hypertext Transfer Protocol) czyniące przeglądanie zasobów globalnej sieci łatwym i intuicyjnym za pośrednictwem każdego rodzaju urządzenia, potężne rozwiązania chmurowe oferujące platformy (PaaS), infrastruktury (IaaS) lub oprogramowanie (SaaS) jako usługi, czy szybka i przenośna bankowość elektroniczna to wybrane powody niesłabnącego zainteresowania rozwiązaniami internetowymi.

Tendencje rozwoju współczesnego oprogramowania warunkują konieczność znalezienia rozwiązania w sytuacjach braku dostępu do sieci Internet. Obejmuje ono metody przechowywania danych lokalnie z użyciem przeglądarki internetowej, sposoby ich replikacji i synchronizacji oraz detekcję połączenia.

Niniejsza praca dokonuje przeglądu wiodących metod lokalnego gromadzenia i przetwarzania danych użytkownika oraz obejmuje stworzenie aplikacji umożliwiającej pracę w trybach online oraz offline wraz z pełną synchronizacją danych pomiędzy klientem a serwerem wraz z bazą danych.

Postaramy się ustalić które z nowoczesnych technologii najlepiej radzą sobie z zachowaniem integralności w razie braku dostępu do sieci Internet.

Struktura niniejszej pracy jest następująca. Rozdział 1 przedstawia koncepcję i cele projektu. W rozdziale 2 dokonano analizy wymagań składowych projektu i przedstawiono wykorzystane technologie oraz paradygmaty. Rozdział 3 przedstawia założenia projektowe poszczególnych komponentów wraz z objaśnieniem wykorzystanych algorytmów. Rozdział 4 poświęcony jest przedstawieniu szczegółów implementacji. W rozdziale 5 zamieszczono krótką instrukcję instalacji i obsługi aplikacji. Rozdział 6 stanowi podsumowanie osiągniętych rezultatów.

\section{Cel pracy}
\label{sec:celPracy}

Celem niniejszej pracy jest przegląd metod lokalnego przechowywania danych z użyciem przeglądarki internetowej oraz projekt i implementacja aplikacji wspierającej tryby offline oraz online w oparciu o technologię HTML5.

\section{Dziedzina problemu}
\label{sec:dziedzinaProblemu}

Przeważająca część tworzonych obecnie aplikacji internetowych nie obsługuje sytuacji braku dostępu do sieci Internet. 

Dynamicznie rozwijające się przeglądarki oferują różne wsparcie dla technologii lokalnego przechowywania danych. Wspomniane mechanizmy stoją na zróżnicowanym poziomie standaryzacji, nierzadko procesy ich rozwijania zostają zawieszone. Różnią się one także stopniem złożoności obsługiwanych struktur danych.

Poważnym ograniczeniem są również limity przesyłanych danych na urządzeniach mobilnych a także koszty transmisji.

Utrudnienia związane z utrzymaniem stałego połączenia internetowego niosą ze sobą problemy z zachowaniem spójności i integralności synchronizowanych danych.

Istotną kwestią jest również dobór metod replikacji bazy danych celem minimalizacji liczby powstających konfliktów i ich poprawna obsługa.
